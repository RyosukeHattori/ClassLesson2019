\documentclass[twocolumn]{jarticle}
\usepackage{jsaiac}
\usepackage{color}
\usepackage[dvipdfmx]{graphicx}
\usepackage{amsmath}
\usepackage{amssymb}
\usepackage{multirow}
\usepackage{url}
\usepackage{bm}
\usepackage{here}

%%
\title{
\jtitle{知的学習システム第3回レポート}
}
%%英文は以下を使用
%\title{Style file for manuscripts of JSAI 20XX}

%\jaddress{岡本 一志,電気通信大学 大学院情報理工学研究科 情報学専攻,東京都調布市調布ヶ丘1-5-1,Tel.: 042-443-5280,E-mail: kazushi@uec.ac.jp}

\author{%
\jname{1930099 服部 凌典}
}

\affiliate{
\jname{電気通信大学 大学院情報理工学研究科 情報学専攻}
}

%%
%\Vol{28}        %% <-- 28th(変更しないでください)
%\session{0A0-00}%% <-- 講演ID(必須)

%\setcounter{page}{1}
\def\Style{``jsaiac.sty''}
\def\BibTeX{{\rm B\kern-.05em{\sc i\kern-.025em b}\kern-.08em%
 T\kern-.1667em\lower.7ex\hbox{E}\kern-.125emX}}
\def\JBibTeX{\leavevmode\lower .6ex\hbox{J}\kern-0.15em\BibTeX}
\def\LaTeXe{\LaTeX\kern.15em2$_{\textstyle\varepsilon}$}

\begin{document}
\maketitle


\section{課題1}
フーリエ級数展開は

\begin{eqnarray}
 f(x) = \frac{a_0}{2} + \sum_{k = 1}^{\infty} (a_k \cos kx + b_k \sin kx)
\end{eqnarray}
である.
ここで,
\begin{eqnarray}
 a_k =  \frac{1}{\pi} \int_{-\pi}^{\pi} f(x) \cos kx \ \d x \\ 
  b_k = \frac{1}{\pi} \int_{-\pi}^{\pi} f(x) \sin kx \ \d x 
\end{eqnarray}
である.
(1)式を展開すると,
\begin{eqnarray}
\begin{split}
 f(x) = \\
 &\ \frac{a_0}{2} + \{ (a_1 \cos x + b_1 \sin x) \\
 &\ + (a_2 \cos 2x + b_2 \sin 2x) + \cdots + \}
 \end{split}
 \end{eqnarray}
 となり,
 \begin{eqnarray}
 f(x) = \frac{a_0}{2} +  \sum_{k = 1}^{\infty} \bm{w}_{k}\bm{\phi_k}(x)
  \end{eqnarray}
  と表せる.
  ただし, $\bm{w}_{k} = (a_k, b_k)$, $\bm{\phi_k}(x) = (\cos kx, \sin kx)^{\mathrm{T}}$とする.
よって,フーリエ級数展開は線形回帰基底モデルである.
 

\section{課題2}
与式の対数尤度関数は
\begin{eqnarray}
\begin{split}
\ln{p(\bm{y}| \bm{w}, \beta) }= \\
&\ln{N(y_1|\bm{w}^{\mathrm{T}} \bm{\phi}(x_1), \beta^{-1})} \\
&+\ln{N(y_2|\bm{w}^{\mathrm{T}} \bm{\phi}(x_2), \beta^{-1})} \\
& + \cdots + \ln{N(y_n|\bm{w}^{\mathrm{T}} \bm{\phi}(x_n), \beta^{-1})}\\
& =  \sum_{k = 1}^{N} \ln{N(y_n|\bm{w}^{\mathrm{T}} \bm{\phi}(x_n), \beta^{-1})}
\end{split}
\end{eqnarray}
である.
また,対数尤度関数の勾配は
\begin{eqnarray}
\begin{split}
 \nabla \ln{p(\bm{y}| \bm{w}, \beta) } = \\
 & \sum_{k = 1}^{N} y_n\bm{\phi}(x_n)^{\mathrm{T}} - \bm{w}^{\mathrm{T}}\left( \sum_{k = 1}^{N}\bm{\phi}(x_n)\bm{\phi}(x_n)^{\mathrm{T}} \right)\\
 &= \bm{0}^{\mathrm{T}}
\end{split}
\end{eqnarray}
である.

\section{課題3}
関数を$f(x) =  10 * (x - 0.5)^2$と定め,
サンプルル点$N = \{100,1000,10000\}$, 基底関数の個数を$M = \{2, 4, 8\}$, $s = 2$, $\beta^{-1} = 0.7$として実験する.
評価方法として平均二乗誤差(MSE)を用いる.
表1にそれぞれのサンプルでのMSEを示す.
表1から,mseはデータセットの個数が多くなるにつれて低くなり,元の関数に回帰モデルが近づくことを確認している.
一方で,基底関数の個数によるmseの変化は見受けられなかった.




\begin{table}[t]
  \begin{center}
      \caption{NとMに対応するMSE}
    \label{table1}
\begin{tabular}{llll}
  & 10   & 100  & 1000 \\
2 & 0.85 & 0.63 & 0.62 \\
4 & 0.85 & 0.63 & 0.62 \\
8 & 0.85 & 0.63 & 0.62
\end{tabular}
\end{center}
\end{table}


\end{document}
